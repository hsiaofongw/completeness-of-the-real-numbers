\documentclass{ctexart}
\usepackage[
    a4paper,
    top=0.8in,
    left=0.8in,
    right=0.8in,
    bottom=1in
]{geometry}
\usepackage{amsmath}
\usepackage{amssymb}
\usepackage{float}
\usepackage{booktabs}
\usepackage{amsthm}
\usepackage{float}
\usepackage{IEEEtrantools}
\pagestyle{plain}

\theoremstyle{definition}
\newtheorem{definition}{定义}

\theoremstyle{definition}
\newtheorem{example}{例}

\theoremstyle{plain}
\newtheorem{prop}{命题}

\theoremstyle{plain}
\newtheorem{theorem}{定理}

\theoremstyle{plain}
\newtheorem{lemma}{引理}

\theoremstyle{definition}
\newtheorem{property}{性质}

\DeclareMathOperator*{\argminA}{\mathrm{arg} \,\mathrm{min}}

\newcommand{\T}{\mathsf{T}}
\newcommand{\nat}{\mathbb{N}^\star}

\begin{document}

\begin{theorem}[单调有界定理]
    单调且有界的数列一定有极限.
\end{theorem}

\begin{theorem}[闭区间套定理]
设$I_n=[a_n,b_n]\,(n\in\mathbb{N}^\star)$,并且$I_{n} \supset I_{n+1}, n \in \nat$,如果这一列区间的长度$|I_n|=b_n-a_n\to 0,\,(n\to\infty)$,那么交集$\displaystyle\bigcap_{n=1}^\infty I_n$含有唯一的一点.
\end{theorem}

\begin{theorem}[列紧性定理]
从任意有界的数列中必可选出一个收敛的子列.
\end{theorem}

\begin{theorem}[柯西收敛定理]
一个数列收敛的充分必要条件是它是柯西列.
\end{theorem}

\begin{theorem}[确界原理]
非空的有上界的集合必有上确界;非空的有下界的集合必有下确界.
\end{theorem}

\begin{theorem}[紧致性定理]
设$[a,b]$是一个有限闭区间,并且它有一个开覆盖$\{ I_\lambda \}$,那么从这个开区间族中必可选出有限个成员来,这有限个开区间所组成的族仍是$[a,b]$的开覆盖.
\end{theorem}

\begin{proof}
$1 \implies 2$. 

\noindent 由$I_n$是闭区间以及对任意$n \in \nat$都有$I_n \supset I_{n+1}$推出$a_n \leq a_{n+1}, b_{n} \geq b_{n+1}$,又根据$a_1 \leq a_n < b_n \leq b_1$推出数列$\{a_n\}, \{b_n\}$都有界,所以数列$\{a_n\}$和数列$\{b_n\}$都存在极限,分别记为$a$和$b$.

\noindent 由于$a_n < b_n$,根据极限的保号性,得$a \leq b$,又根据数列$\{a_n\}, \{b_n\}$的单调性,得
\begin{equation}
    a_n < a \leq b < b_n
\end{equation}
对上述不等式做变形得
\begin{equation}
    0 < b - a < b_n - a < b_n - a_n
\end{equation}
有根据$b_n - a_n \to 0, \, (n \to \infty)$的事实得$b-a=0$,即$b=a$.这个$a$是唯一的,否则如果有$c\neq a$并且$a_n < c < b_n, \, (n \in \nat)$,那么可同理推出$\displaystyle \lim_{n\to\infty} a_n = \lim_{n\to\infty} b_n = c$,由于数列只可能存在唯一的极限,所以$c=a$,这与$c\neq a$的假设矛盾,所以不存在$c \neq a$对任意$n\in\nat$都满足$a_n<c<b_n$.

\noindent 综上所述,我们得
\begin{equation}
    a \in \bigcap_{n=1}^\infty I_n
\end{equation}
并且$a$唯一.
\end{proof}

\begin{proof}
$1 \implies 3$. 令
\begin{equation}
    % i_1 = \argminA_{i \in \nat} \, \{ \forall m \in \nat, a_i \geq a_m \}
    i_1 = \argminA_{\substack{i \in \nat, \\ \forall m \in \nat, a_i \geq a_m}} \, i 
\end{equation}
再令
\begin{equation}
    i_{n+1} = \argminA_{\substack{i \in \nat \\ i > i_n \\ \forall m \in \nat, a_i \geq a_m}} \, i
\end{equation}
如果存在无限多个这样的$i_n$,那么$\{ a_{i_n}\}$可看做是$\{a_n\}$的一个子列并且它显然是单调递减的,由于$\{a_n\}$有界,所以它的任意子列都有界,所以$\{a_{i_n}\}$有界,由单调有界定理,子列$\{a_{i_n}\}$是收敛的.

\noindent 如果只存在有限多个这样的$i_n$,记$i_M, \, M \in \nat$是所有这些$i_n$的最后一项,那么存在一个最小的$j_1 \in \nat$,使得$a_{i_m} < a_{j_1}$,并且还存在$j_2$,满足$j_2 \in \nat, j_2 > j_1$使得$a_{j_1} < a_{j_2}$,一定可以找到无穷多个这样的$j_n$,使得$a_{j_1} < a_{j_2} < a_{j_3} < \cdots$,于是按照这样的方法构造的$\{a_{j_n}\}$是$\{a_n \}$的一个递增子列,显然$\{a_{j_n}\}$有界,于是$\{a_{j_n}\}$收敛.
\end{proof}

\begin{proof}
$3 \implies 4$.

\noindent 先证必要性.设数列$\{a_n\}$的极限是$a$.于是按照极限的定义,对任意$\epsilon >0$,存在$N \in \nat$,使得当$n > N$时,有
\begin{equation}
    |a_n - a| \leq \frac{1}{2} \epsilon
\end{equation}
设$m \in \nat, m > n$,则当$n > N$时也有$m > N$,于是
\begin{equation}
    |a_m - a| \leq \frac{1}{2} \epsilon
\end{equation}
于是
\begin{equation}
    |a_n - a_m| \leq |a_n - a| + |a_m - a| \leq \frac{1}{2} \epsilon + \frac{1}{2} \epsilon < \epsilon
\end{equation}
于是必要性成立.

\noindent 再证充分性.设$\{a_n\}$是一个柯西列.于是,对任意$\epsilon > 0$,都存在$N \in \nat$,使得当$n \geq N$时,对任意$p \in \nat$都有
\begin{equation}
    |a_{n+p} - a_n| < \epsilon
\end{equation}
取特殊的$\epsilon_0 = 1 > 0$,自然也存在$N_0 \in \nat$,使得对任意$p \in \nat$都有
\begin{equation}
    |a_{N_0+p} - a_{N_0}| < \epsilon_0 = 1
    \label{ieq:hasrange}
\end{equation}
于是根据不等式(\ref{ieq:hasrange})易知数列$\{a_n\}$是有界的.有界数列必有收敛子列,可设$\{a_{i_n}\}$是数列$\{a_n\}$的一个收敛子列,设$\{a_{i_n}\}$的极限为$a$,我们来证$a$也是$\{a_n\}$的极限.由于$\{a_{i_n}\}$收敛,所以,对任意$\epsilon > 0$,都存在$N_1(\epsilon) \in \nat$,使得只要$n > N_1$,就有
\begin{equation}
    |a_{i_n}-a|<\frac{1}{2}\epsilon
\end{equation}
又由于$\{a_n\}$是柯西列,所以存在$N_2(\epsilon) \in \nat$,当$n > N_2$时有
\begin{equation}
    |a_{n}-a_{i_n}|<\frac{1}{2}\epsilon
\end{equation}
于是
\begin{equation}
    |a_n - a| \leq |a_{i_n}-a|+|a_n-a_{i_n}| < \frac{1}{2}\epsilon+\frac{1}{2}\epsilon = \epsilon
\end{equation}
从而$\{ a_n \}$收敛于$a$.充分性得证.
\end{proof}
\begin{proof}
$2 \implies 5$.

\noindent 设$c_1$是集合$\{a_n:n\in\nat\}$的一个上界,任取$b_1 \in \{a_n:n\in\nat\}$,令$I_1 = [a_1, b_1]$成为一个闭区间,首先考察闭区间$[\displaystyle\frac{b_1+c_1}{2}, c_1]$是否有$\{a_n\}$中的元素,如果有,那么令$b_2 = \displaystyle \frac{b_1+c_1}{2}, c_2 = c_1$,否则闭区间$[b_1, \displaystyle\frac{b_1+c_1}{2}]$中就一定有$\{a_n\}$的元素,那么就令$b_2=b_1, c_2=\displaystyle\frac{b_1+c_1}{2}$,最后令$I_2 = [a_2, b_2]$.从闭区间$I_n$按类似的步骤构造闭区间$I_{n+1}$:首先考察闭区间$[\displaystyle\frac{b_n+c_n}{2},c_n]$中是否有$\{a_n\}$中的元素,如果有,则令$b_{n+1}=\displaystyle\frac{b_n+c_n}{2}, c_{n+1}=c_n$,否则令$b_{n+1}=b_n,c_{n+1}=\displaystyle\frac{b_n+c_n}{2}$,并且令$I_{n+1} = [b_{n+1},c_{n+1}]$.易知$I_n \supset I_{n+1}, \forall n \in \nat$,并且随着$n\to\infty$,有$|I_n| = |I_1|\displaystyle\frac{1}{2^{n-1}} \to 0$.由闭区间套定理,存在唯一的数$b$满足$b \in \displaystyle\bigcap_{n=1}^\infty I_n$,并且有$\displaystyle\lim_{n\to\infty}b_n=\lim_{n\to\infty}c_n=b$.

\noindent 我们来证明这个$b$正是$\{a_n\}$的上确界,首先要证明的是$b$是$\{a_n\}$的一个上界.假如存在$n_0 \in \nat$使得$a_{n_0} \in \{a_n\}$并且$a_{n_0} > b$,那么令$\epsilon_0 = a_{n_0}-b$,于是存在$N_0 \in \nat$,使得当$n > N_0$时,
\begin{equation}
    |c_n - b| = c_n - b < \epsilon_0
\end{equation}
也就是
\begin{equation}
    c_n - b < a_{n_0} - b
\end{equation}
也就是
\begin{equation}
    c_n < a_{n_0}
\end{equation}
这显然是不可能的,因为由这一系列闭区间$I_1,I_2,I_3,\cdots$的构造过程知,不可能存在$\{a_n\}$中的任何一个元素大于$c_n$.所以,对任意$x \in \{a_n\}$,都有$x \leq b$,从而$b$是$\{a_n\}$的一个上界.

\noindent 现在要证明$b$是$\{a_n\}$的上确界.由于$b$同时是数列$\{b_n\}$与数列$\{c_n\}$的极限,所以,对任意正数$\epsilon >0$,存在$N_1 \in \nat$,使得当$n > N_1$时,有
\begin{equation}
    |b_n-b| = b-b_n < \frac{1}{2}\epsilon
\end{equation}
同时,也存在$N_2 \in \nat$,使得当$n >N_2$时,有
\begin{equation}
    |c_n-b|=c_n-b < \frac{1}{2}\epsilon
\end{equation}
由于对任意$n \in \nat$,闭区间$I_n$都含有$\{a_n\}$中的元素,所以,存在$x_\epsilon \in \{a_n\}$,满足$x_\epsilon \in I_n$,也就是
\begin{equation}
    b_n < x_\epsilon < c_n
\end{equation}
于是当$n > \max \{N_1, N_2\}$时$b -b_n < \displaystyle\frac{1}{2}\epsilon$和$c_n-b<\displaystyle\frac{1}{2}\epsilon$同时成立,并且据此可得
\begin{equation}
    |x_\epsilon - b| \leq |c_n-b_n|=c_n-b_n=b-b_n+c_n-b<\frac{1}{2}\epsilon+\frac{1}{2}\epsilon = \epsilon
\end{equation}
也就是
\begin{equation}
    b - x_\epsilon < \epsilon
\end{equation}
这就证明了$b$就是$\{a_n\}$的上确界.可用类似的二分法,从$\{a_n\}$的任何一个下界和$\{a_n\}$的任何一个元素开始,构造一系列的闭区间$I_1,I_2,I_3,\cdots$构成一个闭区间套,然后用类似的推理证明该闭区间套的唯一交点就是下确界.
\end{proof}

\begin{proof}
$2 \implies 6$. 设$\{I_\lambda : \lambda \in \Lambda \}$是闭区间$[a,b]$的一个开覆盖,于是对任意$x \in [a,b]$,都存在$\lambda(x) \in \Lambda$,使得$x \in I_{\lambda(x)}$.

\noindent 采用反证法,设闭区间$[a,b]$不能够被$\{I_\lambda\}$的有限个开区间组成的开区间簇所覆盖,也就是,任意有限多个$\{I_\lambda\}$中的开区间组成的并集都不能够覆盖$[a,b]$.也就是说,假如$\Lambda_1$是$\Lambda$的一个有限子集,那么一定存在$x_0 \in \{ a_n \}$,满足$x_0 \not\in \displaystyle\bigcup_{\lambda \in \Lambda_1} I_\lambda$.

\noindent 令$a_1 = a, b_1 = b$,令$S_1 = [a_1,b_1]$成为一个闭区间,显然$S_1$不能够被$\{I_\lambda\}$中有限多个开集所覆盖,令$S_2(A) = \displaystyle [a_1, \frac{a_1+b_1}{2}]$,再令$S_2(B) = [\displaystyle\frac{a_1+b_2}{2},b_1]$,在$S_2(A)$和$S_2(B)$这两个闭区间中,必定有一个不能够被$\{I_\lambda\}$的有限个开区间所覆盖,如果$S_2(A)$不能够被$\{I_\lambda\}$有限个开区间所覆盖,就令$a_2 = a_1, b_2 = \displaystyle\frac{a_1+a_2}{2}$,否则令$a_2=\displaystyle\frac{a_1+a_2}{2}, b_2=b_1$,然后令$S_2 = [a_2,b_2]$.显然$S_2$不能够被$\{I_\lambda\}$中有限多个开区间所覆盖.对于任意的$n \in \nat$,按照这样的方法从$S_n$构造出$S_{n+1}$:如果$[a_n,\displaystyle\frac{a_n+b_n}{2}]$不能够被$\{I_\lambda\}$有限个开区间覆盖,就令$a_{n+1}=a_n, b_{n+1} =\displaystyle\frac{a_n+b_n}{2}$,否则令$a_{n+1}=\displaystyle\frac{a_n+b_n}{2},b_{n+1}=b_n$,然后令$S_{n+1}=[a_{n+1},b_{n+1}]$.易知,按照这种方法构造出的一系列闭区间$S_1,S_2,S_3,\cdots$中的每一个都不能够被$\{I_\lambda\}$中的有限个开区间所覆盖.并且,当$n \to \infty$时,第$n$个闭区间$S_n$的长度$|S_n| = |S_1|\displaystyle\frac{1}{2^{n-1}}\to 0$,以及,显然,对所有$n \in \nat$,都有$a_n \leq a_{n+1}$和$b_{n+1} \leq b_n$,从而对每一个$n \in \nat$都有$S_n \supset S_{n+1}$.那么按照闭区间套定理就有$\displaystyle \lim_{n\to\infty} a_n = \lim_{n\to\infty}b_n = \eta$,以及$\eta \in \displaystyle\bigcap_{n=1}^\infty S_n$.

\noindent 显然$\eta \in [a,b]$,于是存在$\lambda_1 \in \Lambda$,使得$\eta \in I_{\lambda_1}$,设$I_{\lambda_1} = (\alpha,\beta)$,令$\epsilon_0 = \min \{\beta-\eta,\eta-\alpha \}$,于是存在$N_1 \in \nat$,使得当$n > N_1$时,
\begin{equation}
    |a_n - \eta| = \eta - a_n < \epsilon_0
\end{equation}
以及存在$N_2 \in \nat$使得当$n > N_2$时
\begin{equation}
    |b_n - \eta| = b_n - \eta < \epsilon_0
\end{equation}
于是当$n > \max \{N_1, N_2\}$时,同时有
\begin{equation}
    \eta - a_n < \epsilon_0 \leq \eta - \alpha, \; b_n - \eta < \epsilon_0 \leq \beta - \eta
\end{equation}
整理得
\begin{equation}
    \alpha < a_n, \; b_n < \beta
\end{equation}
这说明$S_n = [a_n, b_n] \subset (\alpha, \beta) = I_{\lambda_1}$,也就是$S_n$被$\{I_\lambda\}$中有限多个开区间所覆盖,然而根据$S_n$的构造过程我们知道任何一个$S_n$都不被$\{I_\lambda\}$中的有限多个开区间覆盖,于是导出了矛盾.
\end{proof}
\begin{proof}
$3 \implies 1$.
设$\{ a_n \}$是一个单调递增且有界的数列,根据聚点定理,它存在一个收敛子列$\{ a_{i_n} \}$,设这个收敛子列收敛到$a$,那么对任意$\epsilon > 0$,存在$N \in \nat$,使得
\begin{equation}
    0 < a - a_{i_N} < \epsilon
\end{equation}
取$M = i_N$,那么当$n \geq M$时,根据单调性,有$a_n \geq a_M = a_{i_N}$,从而
\begin{equation}
    0 < a - a_{n} \leq a - a_{i_N} < \epsilon
\end{equation}
这就证明了$\{ a_n \}$的极限是$a$,从而$\{ a_n \}$收敛.
\end{proof}

\begin{proof}
$5 \implies 1$.

\noindent 设$\{a_n\}$单调且有界,设$\{a_n\}$是单调递增的,并且根据确界原理,$\{a_n\}$有上确界,设$l$是$\{a_n\}$的上确界.

\noindent 从而对任意$\epsilon > 0$,存在$\{a_n\}$中的元素$x \in \{a_n\}$,不妨记$x$为$x=a_{N(\epsilon)}, N(\epsilon) \in \nat$,使得
\begin{equation}
    l - a_{N(\epsilon)} < \epsilon
\end{equation}
根据$\{a_n\}$的单调性,我们知道,当$n > N(\epsilon)$时,会有$a_n > a_{N(\epsilon)}$,也就是
\begin{equation}
    l - a_n < l - a_{N(\epsilon)} < \epsilon
\end{equation}
也就是
\begin{equation}
    |a_n - l| < \epsilon
\end{equation}
这就证明了$l$是$\{a_n\}$的极限.从而$\{a_n\}$是收敛的.
\end{proof}

\begin{proof}
$2 \implies 3$.

\noindent 设$\{a_n\}$是一个有界数列,如果说$\{a_n\}$中只有有限多个不同的项,那么命题显然成立,证明到此结束;否则就:我们假设$\{a_n\}$中有无穷多个不相等的项,也就是说:集合$\{a_n:n\in\nat\}$中有无穷多个元素.

\noindent 取$b_1$为$\{a_n\}$的一个下界,取$c_1$为$\{a_n\}$的一个上界.于是闭区间$I_1 = [b_1,c_1]$包含了$\{a_n\}$的全部元素,也就是$\forall n \in \nat, a_n \in I_1$.如果闭区间$[b_1,\displaystyle\frac{b_1+c_1}{2}]$包含了$\{a_n\}$中无穷多个元素,那么就令$b_2 = b_1,c_2=\displaystyle\frac{b_1+c_1}{2}]$,否则闭区间$[\displaystyle\frac{b_1+c_1}{2},c_1]$就一定包含了$\{a_n\}$中无穷多个元素,那么令$b_2=\displaystyle\frac{b_1+c_1}{2},c_2=c_1$,然后令$I_2=[b_2,c_2]$;按照类似的方法从闭区间$I_n$构造$I_{n+1}$:检查闭区间$[b_n,\displaystyle\frac{b_n+c_n}{2}]$是否包含$\{a_n\}$中的无穷多个元素,如果是那么就令$b_{n+1}=b_n,c_{n+1}=\displaystyle\frac{b_n+c_n}{2}$,否则令$b_{n+1}=\displaystyle\frac{b_n+c_n}{2},c_{n+1}=c_n$,再令$I_{n+1}=[b_{n+1},c_{n+1}]$.容易证明,这一系列闭区间$I_1,I_2,I_3,\cdots$具有这些性质:1)$I_n \supset I_{n-1}, \forall n \in \nat$,2)第$n$个这样的闭区间的长度随着$n$趋于无穷而趋于零,也就是当$n\to\infty$时$|I_n|=|I_1|\displaystyle\frac{1}{2^{n-1}}\to 0$.3)每一个$I_n, n\in\nat$都包含了$\{a_n\}$中无穷多个元素.

\noindent 为了构造出一个收敛的子列,我们令
\begin{equation}
    i_1 = \argminA_{i \in \nat, a_i \in I_1} \, \{ i \}
\end{equation}
换句话说,$i_1$表示所有能够使得$a_{i} \in I_1$成立的$i \in \nat$中的最小的那一个.对于$m \in \nat, m \geq 2$,我们令
\begin{equation}
    i_m = \argminA_{\substack{i \in \nat, \\ a_i \in I_m, \\ i > i_{m-1} }} \, \{ i \}
\end{equation}
也就是说,$i_m$总是从$i_{m-1}+1$开始寻找,直到找到某个$a_{i_m} \in I_m$.由于每个$I_m$中都有无穷多个$\{a_n\}$的元素,所以这样的$i_1, i_2, i_3, \cdots$总是可以找到的.

\noindent 显然,$1 \leq i_1 < i_2 < i_3 < \cdots$,因此$\{a_{i_n}\}$是$\{a_n\}$的一个子列,下面我们就要来证明这个$\{a_{i_n}\}$收敛.从$i_m$的构造过程可知,对每一个$m \in \nat$,都有
\begin{equation}
    a_{i_m} \in I_m
\end{equation}
由于$I_m$是一个闭区间,$I_m = [b_m, c_m]$,所以上式写成
\begin{equation}
    b_m \leq a_{i_m} \leq c_m
\end{equation}
由于所有的$I_m$构成一个闭区间套$\{ I_m: m \in \nat \}$,所以,根据闭区间套定理,存在唯一的数$b$,使得
\begin{equation}
    \lim_{m\to\infty} b_m = \lim_{m\to\infty} c_m = b
\end{equation}
并且
\begin{equation}
    b \in \bigcap_{m=1}^\infty I_m
\end{equation}
所以有$b_m \leq b \leq c_m$,对任意$m \in \nat$.下面我们要证明的是这个$b$就是$\{a_{i_m}\}$的极限.因为$b = \displaystyle\lim_{n\to\infty}b_n=\lim_{n\to\infty}c_n$,所以,对任意$\epsilon > 0$,存在$M_1 \in \nat$,使得当$m \geq M_1$时
\begin{equation}
    a_{i_m} - b_m < \frac{1}{2}\epsilon
\end{equation}
并且,存在$M_2 \in \nat$,使得当$m \geq M_2$时
\begin{equation}
    c_m - a_{i_m} < \frac{1}{2}\epsilon
\end{equation}
那么当$m \geq \max \{ M_1, M_2 \}$时,上述两个不等式会同时成立,利用这个性质,得到
\begin{equation}
    |a_{i_m} - b| < c_m - b_m = c_m - a_{i_m} + a_{i_m} - b_m < \frac{1}{2}\epsilon+\frac{1}{2}\epsilon = \epsilon
\end{equation}
这说明$\{a_{i_m}\}$的极限是$b$,因此$\{a_{i_m}\}$收敛,因此$\{a_n\}$存在收敛子列.
\end{proof}

\begin{proof}
$4 \implies 2$.
设$I_n = [a_n,b_n]\,(n \in \nat)$,并且当$n\to\infty$时,$I_n \supset I_{n+1}, \forall n \in \nat$,$|I_n| = b_n - a_n \to 0$,我们要来证明存在唯一一点$a$属于交集$\displaystyle\bigcap_{n=1}^\infty I_n$.

\noindent 先证存在性.

\noindent 由于$b_n - a_n \to 0$(当$n\to\infty$),所以,对任意$\epsilon > 0$,存在$N \in \nat$,当$n \geq N$的时候,有
\begin{equation}
    b_n - a_n < \epsilon
\end{equation}
设$p \in \nat$是任意正整数,那么由于$I_n \supset I_{n+1}, \forall n \in \nat$,所以,显然有$I_n \supset I_{n+p}$,由于$I_n, I_{n+p}$都是闭区间,不妨设$I_{n+p} = [a_{n+p},b_{n+p}]$,要使得$I_n \supset I_{n+p}$,就必须要有$a_n \leq a_{n+p} < b_{n+p} \leq b_n$,由此可得
\begin{equation}
    |a_{n+p} - a_n | = a_{n+p} - a_n < b_n - a_n < \epsilon
\end{equation}
以及
\begin{equation}
    |b_{n+p} - b_n | = b_n - b_{n+p} < b_n - a_n < \epsilon
\end{equation}
由柯西列的定义可知数列$\{a_n\}$和$\{b_n\}$都是柯西列,因此数列$\{a_n\}$和数列$\{b_n\}$都收敛.又因为$b_n - a_n < \epsilon$,所以$\displaystyle\lim_{n\to\infty}(b_n-a_n)=0$,所以数列$\{a_n\}$和数列$\{b_n\}$都收敛到同一个极限,不妨记为$a$.

\noindent 由于$I_n \supset I_{n+1},\forall n \in \nat$,所以$\{a_n\}$是单调递增的,并且$\{b_n\}$是单调递减的,又因为$\displaystyle\lim_{n\to\infty}a_n=a$和$\displaystyle\lim_{n\to\infty}b_n=a$,所以,对任意$n \in \nat$恒有$a_n < a < b_n$,如此一来就证明了$\{a_n\}$和$\{b_n\}$收敛,并且它们的共同极限$a$是这一系列闭区间的交集的公共元素.也就是$a \in \displaystyle\bigcap_{n=1}^\infty I_n$.于是存在性得证.

\noindent 再证唯一性.

\noindent 设存在$c \in \mathbb{R}$满足$c \in \displaystyle\bigcap_{n=1}^\infty I_n$并且$c \neq a$.不妨设$c < a$,于是令$\epsilon_0 = a - c$,那么$\epsilon_0 > 0$,于是存在$N(\epsilon_0) \in \nat$,使得当$n \geq N(\epsilon_0)$的时候,
\begin{equation}
    a - a_n < \epsilon_0 = a - c
\end{equation}
推出$a_n > c$,但是由于$c \in \displaystyle\bigcap_{n=1}^\infty I_n$,所以这是不可能的.所以一定不存在这样的$c$既是这些$I_n$的公共元素又满足$c \neq a$.这就证明了唯一性.
\end{proof}

\begin{proof}
$4 \implies 1$.
设$\{ a_n \}$是一个单调递增并且有界的数列,用反证法,假设$\{ a_n \}$发散,那么根据柯西收敛准则的充分性,$\{ a_n \}$不是基本列,于是存在$\epsilon_0 > 0$,对任意$N \in \nat$,都存在$n > m > N$,使得$a_n - a_m > \epsilon_0$,取$N_1 = 1$,存在$n_1 > m_1 > N_1$,使得
\begin{equation}
    a_{n_1} - a_{m_1} \geq \epsilon_0
\end{equation}
再取$N_2 = n_1$,存在$n_2 > m_2 > N_2$,使得
\begin{equation}
    a_{n_2} - a_{m_2} \geq \epsilon_0
\end{equation}
再取$N_3 = n_2$,存在$n_3 > m_3 > N_3$,使得
\begin{equation}
    a_{n_3} - a_{m_3} \geq \epsilon_0
\end{equation}
类似地,我们可以找到一系列的下标$n_1,n_2,n_3,\cdots,m_1,m_2,m_3,\cdots$,使得
\begin{equation}
    a_{n_i} - a_{m_i} \geq \epsilon_0, \; (i = 1,2,3,\cdots)
\end{equation}
并且根据$m_1 < n_1 < m_2 < n_2 < m_3 < n_3 < \cdots$,以及数列$\{ a_n \}$的单调性,我们有
\begin{equation}
    a_{m_1} < a_{n_1} < a_{m_2} < a_{n_2} < a_{m_3} < a_{n_3} < \cdots
\end{equation}
从而,我们有
\begin{equation}
    a_{m_1} + \sum_{i = 1}^t a_{n_t} - a_{m_t} < a_{n_t}, \; (\forall t \in \nat)
\end{equation}
又根据
\begin{equation}
    \sum_{i=1}^{t} a_{n_t} - a_{m_t} \geq \sum_{i=1}^{t} \epsilon_0 = t \epsilon_0
\end{equation}
我们得
\begin{equation}
    a_{n_t} > a_{m_1} + \sum_{i=1}^t a_{n_t} - a_{m_t} \geq t \epsilon_0
\end{equation}
这就推出:当$t \to +\infty$时,会有$a_{n_t} \to +\infty$,也就是说$\{ a_n \}$有一个趋于$+ \infty$的子列,这与$\{ a_n \}$作为一个有界数列的题设不符,矛盾.

\end{proof}

\end{document}